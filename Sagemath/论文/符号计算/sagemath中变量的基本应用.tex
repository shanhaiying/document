%页面设置
\documentclass[a4paper]{article}
\usepackage[top=1in,bottom=1in,left=1in,right=1in]{geometry}
\usepackage{titlesec} %标题位置,有center,raggedleft,raggedright三个选项
%本地化
\usepackage{multicol}
\usepackage{fontspec}
\setmainfont{SimSun}
\XeTeXlinebreaklocale "zh"
\XeTeXlinebreakskip = 0pt plus 1pt minus 0.1pt
%数学
\usepackage{sagetex}
\usepackage{url}
%文档
\begin{document}
\title{数学软件Sagemath中变量的基本应用}
\author{闫鹏}
\date{}
\maketitle
\noindent
\section{摘要}
变量是数学中的基本概念,在初等数学里,变量是一个用来表示值的符号(一般为拉丁字母),该值可以是随意的,也可能是未指定或未定的。在代数运算时,将变量当作明确的数值代入运算中。变量这个概念在微积分中也很重要。一般,一个函数 y = f(x) 会包含两个变量,参数 x 和值 y。这也是“变量”这个名称的由来,当参数“变动”时,值也会相对应地“变动”。另外在更深的数学中,变量也可以只代表某个数据,一般为数字,但也可能为向量、矩阵或函数等数学物件。%\cite{变量} \\

十六、十七世纪,欧洲封建社会开始解体,代之而起的是资本主义社会。由于资本主义工场手工业的繁荣和向机器生产的过渡,以及航海、军事等的发展,促使技术科学和数学急速向前发展。原来的初等数学已经不能满足实践的需要,在数学研究中自然而然地就引入了变量与函数的概念,从此数学进入了变量数学时期。它以笛卡儿的解析几何的建立为起点(1637年),接着是微积分的兴起。%\cite{Sage的发展博客} \\
sage的特点
\begin{enumerate}
\item 软件合集,Sage(用Python和Cython实现的)将所有专用的数学软件集成到一个通用的接口,包括有C 、C++、Fortran和Python编写的大量现成的大型开源数学软件可用,比如Maxima,SymPy,GiNaC,分开学习这些软件将花费大量时间和精力,通过Sagemath提供的统一命令,用户只需要了解Python,即可使用这些软件。
\item 云计算,sage可以运行于本地客户端,利用终端或者浏览器界面使用,也可以在线使用( Sage的在线版本,地址是 sagenb.org 或 https://cloud.sagemath.com),避免下载过G的文件,也有利于移动使用,除了访问官方服务器,用户自己也可以架设本地服务器,供局域网内使用,加快访问和计算速度。
\item 开源软件,其代码开放,提供可供检查的代码增强其权威性,代码也可由用户创建和改造,满足不同用户对于性能和功能的需求。 
\end{enumerate}
\begin{multicols}{2}
sage中定义一个变量可以有两种方式:
第一种是通过赋值的方法定义一个变量,该变量的值是指定的,比如,等式的左边是符号变量的名称,右边是数字或表达式,其值返回给符号变量。Python动态类型的语言,符号变量可以多次赋值,并可以傅以不同类型的值,比如,在sage中可以通过type()得到符号变量的类型,这些类型有。。。。需注意sage的符号变量区分大小写,每个符号变量仅存在于当前的工作表单(引申),如果当前的工作表单被关闭后重开,则应该再次执行赋值。
\begin{sagecommandline}
sage: a1=2
\end{sagecommandline}
\begin{sagecommandline}
sage: a2=3
sage: sum=a1+a2
sage: sum
\end{sagecommandline}
也可以把多个语句放在一行,语句之间使用分号隔开。
\begin{sageexample}
sage: a1=2;a2=3;sum1=a1+a2;sum2=a1-a2
sage: sum1;sum2
\end{sageexample}

更复杂的语句包含不同操作运算符,如下例子
\begin{sageexample}
sage: a1=2
sage: a2=2^2+3*4+(a1>5)
sage: a2 
\end{sageexample}
\end{multicols}
\end{document}
