%页面设置
\documentclass[a4paper]{article}
\usepackage[top=1in,bottom=1in,left=1in,right=1in]{geometry}
%本地化
\usepackage{multicol}
\usepackage{fontspec}
\setmainfont{SimSun}
\XeTeXlinebreaklocale "zh"
\XeTeXlinebreakskip = 0pt plus 1pt minus 0.1pt
\setlength\parindent{0pt} %首行不缩进
%数学
\usepackage{sagetex}
\usepackage{url}
%文档
\begin{document}
\title{数学软件Sagemath中变量的基本应用}
\author{闫鹏}
\date{}
\maketitle
\noindent
\section{摘要}
变量是数学中的基本概念,在初等数学里,变量是一个用来表示值的符号(一般为拉丁字母),该值可以是随意的,也可能是未指定或未定的。在代数运算时,将变量当作明确的数值代入运算中。变量这个概念在微积分中也很重要。一般,一个函数 y = f(x) 会包含两个变量,参数 x 和值 y。这也是“变量”这个名称的由来,当参数“变动”时,值也会相对应地“变动”。另外在更深的数学中,变量也可以只代表某个数据,一般为数字,但也可能为向量、矩阵或函数等数学物件。%\cite{变量} \\

十六、十七世纪,欧洲封建社会开始解体,代之而起的是资本主义社会。由于资本主义工场手工业的繁荣和向机器生产的过渡,以及航海、军事等的发展,促使技术科学和数学急速向前发展。原来的初等数学已经不能满足实践的需要,在数学研究中自然而然地就引入了变量与函数的概念,从此数学进入了变量数学时期。它以笛卡儿的解析几何的建立为起点(1637年),接着是微积分的兴起。%\cite{Sage的发展博客} \\
sage的特点
\begin{enumerate}
\item 软件合集,Sage(用Python和Cython实现的)将所有专用的数学软件集成到一个通用的接口,包括有C 、C++、Fortran和Python编写的大量现成的大型开源数学软件可用,比如Maxima,SymPy,GiNaC,分开学习这些软件将花费大量时间和精力,通过Sagemath提供的统一命令,用户只需要了解Python,即可使用这些软件。
\item 云计算,sage可以运行于本地客户端,利用终端或者浏览器界面使用,也可以在线使用( Sage的在线版本,地址是 sagenb.org 或 https://cloud.sagemath.com),避免下载过G的文件,也有利于移动使用,除了访问官方服务器,用户自己也可以架设本地服务器,供局域网内使用,加快访问和计算速度。
\item 开源软件,其代码开放,提供可供检查的代码增强其权威性,代码也可由用户创建和改造,满足不同用户对于性能和功能的需求。 
\end{enumerate}
\begin{multicols}{2}
sage中定义一个变量可以有两种方式:
%第一种方式
第一种是通过赋值的方法定义一个变量,该变量的值是指定的,比如,等式的左边是符号变量的名称,右边是数字或表达式,其值返回给符号变量。Python动态类型的语言,符号变量可以多次赋值,并可以傅以不同类型的值,比如,在sage中可以通过type()得到符号变量的类型,这些类型有。。。。需注意sage的符号变量区分大小写,每个符号变量仅存在于当前的工作表单(引申),如果当前的工作表单被关闭后重开,则应该再次执行赋值。
\begin{sageblock}
sage: a1=2
sage: a2=3
sage: sum=a1+a2
\end{sageblock}
\begin{sageblock}
sage: sum
\end{sageblock}
也可以把多个语句放在一行,语句之间使用分号隔开。
\begin{sageblock}
sage: a1=2;a2=3;sum1=a1+a2;sum2=a1-a2
\end{sageblock}
\begin{sageblock}
sage: sum1;sum2
\end{sageblock}
更复杂的语句包含不同操作运算符,如下例子
\begin{sageblock}
sage: a1=2
sage: a2=2^2+3*4+(a1>5)
\end{sageblock}

运算符执行先后顺序如下: \\

\begin{tabular}{lc}
	运算符	 &  	说明\\ \hline
	or		 &  逻辑或\\ \hline
	and	& 逻辑与	  \\ \hline
	not		 &  逻辑非\\ \hline
	in, not in		 & 隶属关系 \\ \hline
	is, is not		 &  类型检测\\ \hline
	>, <=, >, >=, ==, !=, <>	 & 比较 \\ \hline
	+, -	 &  加、减\\ \hline
	*, /, \percent		 & 乘、除、余 \\ \hline
	**, \^{}	 & 指数 \\
	\hline 
	\end{tabular} 
条件运算符(==,<>,!=,<,<=,>,>=),其返回值是布林值,即True或者False,以数值表示为1、0,布林运算符(not,and,or或与非运算符),其返回值是True或者False,以数值表示则为1、0。

\begin{sageblock}
sage: a2 
\end{sageblock} 
%第二种方式
第二种是使用var函数定义未赋值的符号变量,未赋值的符号变量在其定义区间内的值是任意的,默认情况下,使用var函数定义的变量是复数范围内的,注意在Sage中x是已经定义好的未赋值符号变量,可以直接使用。同时Sage定义了一个变量``\_"用来接受上一次运算的结果。
\begin{sageblock}
var('y,z') %定义y,z两个符号变量
z=x+y
_+y
\end{sageblock}

reset函数可以取消定义,用法$reset('y,z')$,但变量x取消后依然存在。 \\
含有符号变量的表达式又叫符号表达式,如$z=x+y$,$x+y$即为符号表达式(Symbolic Expression),我们可以对符号表达式中的变量赋值:
\begin{sageblock}
z(x=2,y=3)
\end{sageblock}
如果符号表达式只有一个变量,则可以直接赋值。
\begin{sageblock}
z=2x
z(5)
\end{sageblock}

不同符号表达式可以进行运算,比如
\begin{sageblock}
z=(x-y)*(x+y)
\end{sageblock}

利用expand()函数我们可以对其展开
\begin{sageblock}
expand(z)
\end{sageblock}

也可以通过Python函数调用的方法来展开
\begin{sageblock}
z.expand()
\end{sageblock}

结果为$x^2-y^2$,函数或函数调用并不会影响$ z $本身的属性,此时z依然为$(x+y)(x-y)$,现在我们定义$f=z.expand()$。
\begin{sageblock}
	sage: f=z.expand()
\end{sageblock}
利用$factor()$函数可以求取公因式$factor(f)$或$f.factor$,factor函数同样可以分解数字。
\begin{sageblock}
	sage: factor(20)
\end{sageblock}

对于分式分解,我们可以利用factor函数求出公因式,然后自行分解,也可以利用``partial\_fraction"函数来分解。
\begin{sageblock}
	sage: z=1/(x^2-1)
	sage: z.partial_fraction()
\end{sageblock}
现在我们来看两个表达式
\begin{sageblock}
	sage: z1=x^2+3*x+1
	sage: z2=x^2+x+1
\end{sageblock}
多项式所在的环影响它的性质。因此对上面两个表达式进行因式分解,得到本身。可以发现,z1==0在实数范围内有解,z2==0在复数范围内有解,即这两个多项式可以通过先求其解来进行分解,当然我们也可以通过指定多项式其所在的“环”,然后用factor函数来分解 %
最简单的方法是
\begin{sageblock}
	sage: R.<x>=CC[]
\end{sageblock}
这里R定义一个环,x定义一个域,域的区间为CC(复数域)QQ,RR等等
\begin{sageblock}
	sage: z1.factor();z2.factor()
\end{sageblock}
对于z1,我们也可以仅将其变量定义在实数域中。
\begin{sageblock}
	sage: R.<x>=RR[]
\end{sageblock}
需要指出,环由变量确定。相同域上的同一变量得到的环是等价的。
\begin{sageblock}
	sage: R.<x>=QQ[];Z.<x>=QQ[]
	sage: R==Z
\end{sageblock}
同样,我们可以直接在定义向量和矩阵时指定其所在的环。
\begin{sageblock}
	sage: v = vector(QQ, (1,2,3))
\end{sageblock}  

矩阵 
\begin{sageblock}
	sage: Z = matrix(ZZ, [[2,0], [0,1]])
\end{sageblock}  
多项式化简 \\
在Sage中多项式化简是自动的
\begin{sageblock}
	sage: var(`y')
	sage: z=x+y-y
	sage: z
\end{sageblock}
使用Python的id函数可以查看变量在内存中的位置
\begin{sageblock}
	sage: id(z);id(x)
\end{sageblock}
可见变量z的保存位置和变量x相同,当你输入一个多项式时,如果其可以简单化简,则是多项式实际是按照化简之后的形式保存的。
如果多项式不能直接化简,我们可以对其手动化简,这时用到simplify系列函数
\begin{sageblock}
	sage: z = (x^2-1)^(1/2)/(x+1)^(1/2);show(z)
\end{sageblock}
可以看出这个式子可以化简,这里用到$simplify_radical$函数
\begin{sageblock}
	sage: z.simplify_radical();show(_)
\end{sageblock}
simplify系列函数有很多个,在Sage中输入simplify,然后按下Tab键即可查看,对于每个函数,在函数名后输入?可以查看该函数的帮助信息,输入??可以查看该函数的源代码。

我们可以通过solve()函数求的方程的解
\begin{sageblock}
	sage: var('a)
\end{sageblock}
求变量x;求变量a
\begin{sageblock}
	sage: solve(x^2==a^2,x);solve(x^2==a^2,a)
\end{sageblock}
结果以列表的形式出现,列表是Python的内建数据类型,形式为[exp1, exp2, exp3,...]
方程组的求解也很容易
\begin{sageblock}
	sage: var('a,b,c')
	sage: solve([2*a+b-c==0,3*b-c==0,a+b==5],a,b,c)
\end{sageblock}
这里需注意,在Python语言中,除法公式$5/2$结果为2,这是因为5和2是整你形,所以其运算结果也是整形,Sage则返回5/2本身,不做改变,通过类型提升我们可以得到5/2的数字解
\begin{sageblock}
	sage: float(5/2)
\end{sageblock}
sage也提供n()函数,使用方法是n(5/2)或5/2.n(),n()中可以指定精度和位数,注意这里的精度并不是精确到小数点后面几位,而是浮点数中尾数的存储位数,默认Sage使用53bits。
n(5/3,prec=30,digits=8)指定5/3的结果精度为5,有8位有效数字

符号函数
在Python中元组通过圆括号中用逗号分割的项目定义。它和列表,字典同属序列,与列表不同,元祖中的对象既不能改变,也不能赋值。Sage引入了数学语法来定义函数,因此我们可以这样定义函数
f(x)=2*x
在这种情况下,括号中的x为函数f的参数,我们可以对其赋值:
\begin{sagecommandline}
	f(2);f(4)....
\end{sagecommandline}
在表达式z=x+y中,实际上我们定义了z(x,y)=x+y
\begin{sagecommandline}
	z(x=2,y=3)
\end{sagecommandline}

我们通常想得到一个函数的图像,使用$plot$函数可以方便的做出二维图形,这里有一个函数$f(x)=x^2-10$
plot(f(x),-10,10),这里-10,10为变量x的取值范围的闭区间,即为坐标轴中x的范围

Sage也能创建三维图像,显示三维图像默认都是调用开源软件包 [Jmol], 它支持使用鼠标旋转和缩放图像(需要Java运行环境)。这里有一个函数f(x,y)=x-sin(y)
\begin{sagecommandline}
	plot3d(f(x,y),(x,-10,10),(y,-10,10))
\end{sagecommandline}
当我们使用$z=plot(x^2,-1,1)$时,这个过程只是定义变量z,并不会将结果显示出来,这里我们用到show函数。
\begin{sagecommandline}
	z=plot(x^2,-1,1);show(z)
\end{sagecommandline}
show函数还可以显示手写格式的数学公式,这时实际是调用latex,上面我们已经用到了show函数。
\begin{sageblock}
	z = (x^2-1)^(1/2)/(x+1)^(1/2);show(z)
\end{sageblock}

也可以把多个图像一起做图:
\begin{sagecommandline}
	f1(x)=x
	f2(x)=x^2-10
	plot([f1(x),f2(x)],-10,10)
\end{sagecommandline}
我们看到f1和f2在[-10,10]的范围内有交点,现在我们求该交点,求交点的方法即是解方程的方法,上文我们用solve函数解代数方程,但是并不是所有的方程都有代数解,在这里我们$find_root$函数在区间内找到它的数值解。
\begin{sageblock}
	f=f1(x)-f2(x)==0
	f.find_root(0,10);f.find_root(-10,0) 
\end{sageblock}
$find_root$每次只会在给定区间内求出一个解,因此结合图像,我们可以细分区间,求出所有的交点。


在代数运算时,将变量当作明确的数值代入运算中。



微积分
变量在高等数学中有着非常重要的应用,以微积分为例
%微分/偏微分,积分,泰勒级数,
使用diff函数可以求函数的导数,微分和偏微分,基本格式为diff(函数,*偏微分变量,*微分阶次)
\begin{sagecommandline}
	var(`y')
	diff(sin(x));diff(sin(x-y),y);diff(sin(x),4);diff(sin(x-y),y,4)
	也可以令z1=sin(x),z2=sin(x-y)使用调用的方法
	z1.diff();z2.diff(y);z1.diff(4);z2,diff(y,4)
\end{sagecommandline}
%泰勒和罗伦斯级数
泰勒级数(Taylor series)用无限项连加式——级数来表示一个函数,这些相加的项由函数在某一点的导数求得。洛朗级数是泰勒级数的推广,它不仅包含了正数次数的项,也包含了负数次数的项。有时无法把函数表示为泰勒级数,但可以表示为洛朗级数。使用talyor函数可以快速的求出函数在某点上的泰勒或洛郎级数,下面是个例子。
\begin{sagecommandline}
	(x - a)^n
	
	"x, a, n" - variable, point, degree
	
	x^2.taylor(x,1,6)
\end{sagecommandline}
函数x^2在x=1点上的六级泰勒级数,如果是-1级级数,则求出的是劳伦斯级数

使用integral函数可以求函数的积分,定积分和不定积分,基本格式是integral(函数,积分变量,*积分范围),对于高阶和多元函数的积分,可以通过叠加使用integral函数,注意在Matlab中int为积分符号,而在Python中int为取整,int(pi)=3.
\begin{sagecommandline}
	var(`y')
	integral(sin(x),x);integral(sin(x-y),y);integral(sin(x),x,0,1);integral(integral(sin(x-y),x,0,1),y,0,1) %手写数学符号
\end{sagecommandline}
现在我们看这样的一个函数int(sin(x)e^-(st))
\begin{sagecommandline}
	assume(s>0)
	integrate(sin(x)*e^(-s*x),x,0,+Infinity)
	1/(s^2 + 1)
\end{sagecommandline}
实际上int(sin(x)e^-(st))是sin(x)的拉普拉斯变换形式,拉普拉斯变换是应用数学中常用的一种积分变换,又名拉氏转换,在Sage中,拉式变换可以使用laplace函数,格式为laplace(f(x),x,s),即对函数f(x)进行拉式变化,变换后的变量为s,拉式变换对应的反变换是inverse_laplace,格式为inverse_laplace(f(s),s,x)。下面是一个例子。
\begin{sagecommandline}
	var(`s')
	laplace(sin(x),x,s)
	inverse_laplace(_,s,x)
\end{sagecommandline}

傅里叶变换是另一种一种线性的积分变换,常在将信号在时域(或空域)和频域之间变换时使用。Sage中并没有直接提供相应的函数支持,但我们可以根据傅里叶函数的形式使用积分来计算,有时候,根据拉普拉斯变换与傅里叶变换的相似性,我们还可以套用拉普拉斯变换来求傅里叶变换。实际上,由于傅里叶变换在物理学和工程学中的广泛应用,有很多专门的文献、资料涉及傅里叶变换,对于Sagemath和傅里叶变换的结合,读者可以参考Computational Fourier Transforms 一书,本文不再涉及,后续我也会通过专门的文章来探讨这个问题。

微分方程包括常微分方程和偏微分方程,两者都可以再分为线性和非线性两类,线性包括齐次和非齐次,使用desolve函数可以解一介和二阶线性微分方程。Solves a 1st or 2nd order linear ODE via maxima. Including IVP and BVP.
我们以desolve的帮助文档中的例子为例。

一阶微分方程
\begin{sagecommandline}
	function('y', x) #定义y(x),即y是变量x的函数
	desolve(diff(y,x) + y - 1, y) #数学形式?,求y
	(_C + e^x)*e^(-x) #结果
\end{sagecommandline}
这里用到了Sage调用Maxima的接口,所以它的输出看上去与其他Sage的输出略有不同。可以通过_.show()来显示出数学形式,?方便阅读。如果指定积分区间,ics - (optional) the initial or boundary conditions则我们可以进一步求出C
\begin{sagecommandline}
	desolve(diff(y,x) + y - 1, y, ics=[10,2]); f
	(e^10 + e^x)*e^(-x)
\end{sagecommandlinf.derivativee}
二阶微分方程
\begin{sagecommandline}
	function('y', x)
	diff(y,x,2) - y == x
	desolve(de, y).show()
	???_K2*e^(-x) + _K1*e^x - x
\end{sagecommandline}
请参考求解微分方程的方法(低阶到高阶的方法统计)http://wiki.sagemath.org/Differential_Equations

Sagemath是一个复杂的数学系统,一方面它包容了70多个成熟的开源数学工具,比如。。。,当我们需要在Sage中显性的调用这些工具时,我们也需要对其有一个初步的了解,另一方面,作为一个纯粹的学术工具,Sage缺乏像matlab中各种专业工具箱,不利于Sage在专业领域的发展,当然中文资料的匮乏也不利于Sagemath在国内的发展。但,Sagemath的开发速度很快?越来越多的人参与到这个开源数学工具的开发和推广上来,现在西方已经有很多人?开始在专业领域应用Sagemath,如果想进一步了解Sage,可以到其官网下载使用,或使用其在线版本,Sagemath目前已经是一个非常成熟的数学系统,在学术界接受程度很高,在很多大学?,国内?有应用,也多次作为演算工具,被各类论文引用,在目前反盗版,推荐国内各类高校部署和推广Sagemath。
\begin{thebibliography}{50}
	\bibitem{变量} \url{http://zh.wikipedia.org/zh-cn/%E8%AE%8A%E6%95%B8}
		\bibitem{Sage} \url{http://en.wikipedia.org/wiki/Sage_(mathematics_software)#cite_note-10}
		\bibitem{Sage for Newbie}
		\bibitem{环} \url{file:///home/yub/%E6%96%87%E6%A1%A3/Sagemath/tutorial/tour_rings.html http://localhost:8080/doc/live/reference/rings/sage/rings/ring.html}
			\bibitem{精度} \url{http://zh.wikipedia.org/zh-cn/%E5%8F%8C%E7%B2%BE%E5%BA%A6%E6%B5%AE%E7%82%B9%E6%95%B0}
				\bibitem{元祖} \url{http://woodpecker.org.cn/abyteofpython_cn/chinese/ch09s03.html}
				\bibitem{做图} \url{file:///home/yub/%E6%96%87%E6%A1%A3/Sagemath/tutorial/tour_algebra.html#section-systems}
					\bibitem{二元数学运算符优先级} \url{file:///home/yub/%E6%96%87%E6%A1%A3/Sagemath/tutorial/appendix.html#section-precedence}
						\bibitem{变量数学} 数学的三个发展时期——变量数学时期(2008-01-24  转自《大科普网》) \url{http://www.pep.com.cn/gzsxb/xszx/czsxkwyd_1/czsxkwydsxgs/201009/t20100929_922925.htm}
						\bibitem{微积分} \url{file:///home/yub/%E6%96%87%E6%A1%A3/Sagemath/tutorial/tour_algebra.html}
							\bibitem{微积分} \url{calculus.pdf}
							\bibitem{Sage工业应用网站} \url{http://vibrationdata.com/python-wiki/index.php?title=Main_Page}
							\bibitem{Sage的发展博客} \url{http://blog.163.com/soft_share@126/blog/static/42983603201312592144105/}
							\bibitem{Sage中定义的常量}
							\bibitem{傅里叶变换} \url{http://zh.wikipedia.org/wiki/%E5%82%85%E9%87%8C%E5%8F%B6%E5%8F%98%E6%8D%A2}
								\bibitem{拉普拉斯变换} \url{http://zh.wikipedia.org/zh-cn/%E6%8B%89%E6%99%AE%E6%8B%89%E6%96%AF%E5%8F%98%E6%8D%A2}
									\bibitem{Sage Notebook服务器地址大全} \url{http://www.ai7.org/wp/html/904.html}
									\bibitem{求解微分方程的方法(低阶到高阶的方法统计)} \url{http://wiki.sagemath.org/Differential_Equations}
									\bibitem{微分方程} \url{http://zh.wikipedia.org/zh-cn/%E5%BE%AE%E5%88%86%E6%96%B9%E7%A8%8B}
										\bibitem{Teaching_with_SAGE} \url{http://wiki.sagemath.org/Teaching_with_SAGE?highlight=%28fourier%29Computational} 
											\bibitem{Computational Fourier Transforms} 
											\bibitem{泰勒级数} \url{http://zh.wikipedia.org/wiki/%E6%B3%B0%E5%8B%92%E7%BA%A7%E6%95%B0}
												\bibitem{洛郎级数} \url{http://zh.wikipedia.org/wiki/%E6%B4%9B%E6%9C%97%E7%BA%A7%E6%95%B0}
												\end{thebibliography}
\end{multicols}
\end{document}
