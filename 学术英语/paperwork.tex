\documentclass[a4paper,10.5pt]{article}
\setlength{\parindent}{0pt} %首行不缩进
\setlength{\baselineskip}{20pt} %行距
%宏包
\usepackage[top=1in,bottom=1in,left=1in,right=1in]{geometry}
\usepackage[english]{babel}
\usepackage{fontspec} %使用非latex字体要使用此包
\setmainfont{Times New Roman} 
%更改大列表环境,小列表环境可以在label中改变
\renewcommand{\theenumi}{\Roman{enumi}.}
\renewcommand{\labelenumi}{\theenumi}
\renewcommand{\theenumii}{\Alph{enumii}.}
\renewcommand{\labelenumii}{\theenumii}
\renewcommand{\theenumiii}{\arabic{enumiii}.}
\renewcommand{\labelenumiii}{\theenumiii}
\renewcommand{\theenumiv}{\alph{enumiv}.}
\renewcommand{\labelenumiv}{\theenumiv}


%文档
\begin{document}
\title{HJD Type Crane PLC transformation in control circuit \\
{\fontspec{SimHei} HJD型克令吊控制电路的PLC改造}}
\author{\fontspec{SimHei} 船海 闫鹏 201430110059}
\date{}
\maketitle

\section*{Abstract}
\textbf{Background}. HJD type crane is widely used in domestic ships. Traditional HJD type crane adopts relays and contactors for secondary circuit control. The coils and contacts are easily burned after long-term use, frequently causing accidents. \\
\textbf{Aims and Approach}. The chief aim of the present work is provide an available way shifting to PLC control and reducing the use of coils and contacts. PLC, namely, programmable logic controller, is a digital computer used for automation and control. PLCs support multiple analogue and digital inputs and outputs arrangements, with extended temperature ranges, immunity to electrical noise, and resistance to vibration and impact. PLC used in this paper is Siemens S7-200.\\
\textbf{Conclusion}. In this paper, The PLC transformation, including basic protection control process, rose and fall process, brake process, low/medium/high-speed process, has been supplied after analyzing the principle of control circuits. There are no failures in experiment after long-term usage. 

\section*{Outline}
Thesis Sentence: The PLC transformations and the method of implementation, including basic protection control process, rose and fall process, brake process, low/medium/high-speed process, list below.
\begin{enumerate}
\item Reconstruction of basic protection control process	
	\begin{enumerate}
\item Overload protection of fan motor and crane motor
		\begin{enumerate}
\item Fan motor overload protection is achieved  by fan thermal relay FR2.
\item Crane motor overload protection is achieved  by crane motor thermal relay FR1.
		\end{enumerate}
\item Motor windings overheating protection is achieved by  motor temperature controller ST.
\item Power supply missing phase and circuit break protection are achieved by  zero-voltage relay KA1.
\item Emergency forced running is achieved by the contactor SB.
	\end{enumerate}
\item Reconstruction of rose and fall process
	\begin{enumerate}
\item Rose process  is achieved by steering control contactor Q0.2.
\item Fall process is  achieved by steering control contactor Q0.3.
\item DC delay time relay is required in reversing at high speed.
	\end{enumerate}
\item Reconstruction of brake process
	\begin{enumerate} 
\item In normal rise or fall state, parking brake   coil pulls in,mechanical parking brake operates.
\item In medium/high gear, DC master switch disconnect, low speed winding connects realizing automatic grade braking.
	\end{enumerate} 
\item Reconstruction of low/medium/high-speed process
	\begin{enumerate}
\item Low-speed process
		\begin{enumerate}
\item The crane will run in operational status by releasing rise or fall  contactor and the braking contactor.
\item The shift from low-speed to medium-speed is achieved by energize medium-speed winding power.
		\end{enumerate}
\item Medium-speed process
	\begin{enumerate}
\item Medium speed contactor is self-locking and  interlocking with  low/high speed contactor.
\item Braking contactor and fan contactor make sure the Motor does not run at medium speed.
	\end{enumerate}
\item High-speed process
	\begin{enumerate}
\item Motor should not run in high speed when heavy loaded.
	\begin{enumerate}
\item The detection of load is achieved  by load relay.
\item The protection of overload is achieved by load contactors.
	\end{enumerate}
\item The diagram of wiring and control logic has been uploaded to PLC 
	\end{enumerate} 
	\end{enumerate}
	\end{enumerate}
	
\section*{References}
\hangafter=1
\setlength{\hangindent}{4em}
Smoczek, J., and J. Szpytko. 2014. “Evolutionary algorithm-based design of a fuzzy TBF predictive model and TSK fuzzy anti-sway crane control system.” \textit{Engineering Applications of Artificial Intelligence} 28:190-200. \par

\hangafter=1
\setlength{\hangindent}{4em}
Kłosiński, Jacek. 2005. “Swing-free stop control of the slewing motion of a mobile crane.”
\textit{Control Engineering Practice} 13:451-460. \par

\hangafter=1
\setlength{\hangindent}{4em}
Kim, Dooroo, and William Singhose. 2010. “Performance studies of human operators driving double-pendulum bridge cranes.” 
\textit{Control Engineering Practice} 18:567-576. \par

\hangafter=1
\setlength{\hangindent}{4em}
Armstrong, N.A., and P.R. 1994. “MooreA distributed control architecture for intelligent crane automation.” 
\textit{Automation in Construction} 3:45-53. \par

\hangafter=1
\setlength{\hangindent}{4em}
Das, S.K., S. Utku, and B.K. Wada. 1990. “Use of reduced basis technique in the inverse dynamics of large space cranes.”
\textit{Computing Systems in Engineering} 1:577-589. \par

\hangafter=1
\setlength{\hangindent}{4em}
Chang Chengyuan, and Kuo-Hung Chiang. 2008. “Fuzzy projection control law and its application to the overhead crane.”
\textit{Mechatronics} 18:607-615. \par

\hangafter=1
\setlength{\hangindent}{4em}
Armstrong, N.A., and P.R. Moore. “A distributed control architecture for intelligent crane automation.”
\textit{Automation in Construction} 3:45-53. \par

\hangafter=1
\setlength{\hangindent}{4em}
Rusinski, Eugeniusz, Zaklina Stamboliska, and Przemysław Moczko. 2013. “Proactive control system of condition of low-speed cement machinery.”
\textit{Automation in Construction} 31:313-324. \par

\hangafter=1
\setlength{\hangindent}{4em}
HORÁČEK, P.. 1995. “MODULAR CONTROL LABORATORY ARCHITECTURE.”
\textit{Advances in Control Education} 253-256.

\section*{Acknowledgments}
The author acknowledge the support of the Shanghai Maritime
University and the college of Merchant Marine. The author appreciate the valuable
help from instructor Mr. Ling, and express his gratitude to all those who helped him
during the writing of this thesis. The content is solely the responsibility of the author and
does not necessarily represent the official views of the Shanghai Maritime University
and Merchant Marine College.
\end{document}
