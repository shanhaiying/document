%% LyX 2.1.2.2 created this file.  For more info, see http://www.lyx.org/.
%% Do not edit unless you really know what you are doing.
\documentclass{ctexart}
\usepackage[a4paper]{geometry}
\geometry{verbose,tmargin=1in,bmargin=1in,lmargin=1in}

\makeatletter
%%%%%%%%%%%%%%%%%%%%%%%%%%%%%% User specified LaTeX commands.
\usepackage{ulem}

\makeatother

\usepackage{xunicode}
\begin{document}

\section{名词解释 }


\subsection{多自由度振动:需要两个或两个以上的独立坐标才能描述其运动的振动系统。P36 }


\subsection{共振:干扰力的频率与自振频率相重合,位移和内力都将无限增加,这种现象称为共振。P17 L9 }


\subsection{局部振动:船体局部结构,如板架、梁、板隔等对于船体所做的附加振动。对应于总振动。P111 p2}


\subsection{声压的概念:有声波作用时,声波的传播实际上是媒质内稠密和稀疏的交替过程,体积元由于声扰动产生的前后压强 逾量,称为声压。P181 }

\subsection{吸声系数:为被吸收声能(包括透射声能)与入射声能之比。P197 }


\subsection{声阻抗率:声场中某位置的声压与该位置的质点速度的比值。P183 }


\subsection{固有振型:质点自由振动即简谐振动时,其每秒振动的弧度或2π秒内振动的次数仅取决于系统固有性质而与运动初 始条件无关,称为系统的固有频率。P5-6 }


\subsection{上层建筑振动:上层建筑整体的纵向振动和上层建筑局部构件的振动。P154}


\subsection{浮筏隔振系统:实际上是多机组双层隔振系统,将船舶主要振源设备通过上层隔振器弹性地安装在一个公共筏体上,然后将公共筏体通过下层的隔振器弹性地安装在船体上。P267-268}


\subsection{叶厚效应:螺旋桨桨叶具有厚度,在流场中运动时,流场中某一点P处压力将随着桨叶的接近和远离该点发生周期性的变化,从而使该流场中个点受到脉动压力。 }


\section{判断题 }


\subsection{工程共振现象时振幅无穷大(✗)P20 }


\subsection{阻尼消耗能量,使振动衰弱(✔)}


\subsection{无阻尼振动系统振动频率比有阻尼的小(✗) 只改变振幅 }


\subsection{舰船的主要噪声源只有柴油机噪声,燃气轮机噪声,轴承 噪声,液力机械噪声,电机噪声五种(✗)P286 }


\subsection{船舶是一种复杂的水上建筑物,其结构及质量分布很不规 则,是等截面的空心梁。(✗)P111 变截面 }


\subsection{船体振动所受到的力有干扰力,弹性恢复力,惯性力和阻 尼力(✔)P111 p4 }


\subsection{节点就是船体总振动时振幅最大的点(✗)P71 图旁边 }


\subsection{结构内阻尼力是因为系统本身结构缺陷而引起的(✗)P169 p3 }


\subsection{系统对初始激励的响应通常称为自由振动(✔) P4 }


\subsection{只有刚度和强度是衡量减振器用途的功能特性的参数(✗) P275 }


\section{填空}


\subsection{振动微分方程为: $m\frac{d^{2}x}{dt^{2}}+c\frac{dx}{dt}+kx=Fsin(wt)$}

方程中第一项代表的意义为:\uline{惯性力},第二项为:uline{黏性阻尼力},方程中第一项代表的意义为:\uline{弹性力},第四项为:\uline{简谐干扰力}。


\subsection{结构的固有频率仅与结构的\uline{质量}和\uline{弹簧刚度}有关,与结构初始的振动状态以及干扰力无关。
P6 }


\subsection{舷外水对船体总振动的影响可分为\uline{重力,阻尼}和\uline{惯性}等三个方面。P115 }


\subsection{内损耗功率通常由\uline{结构阻尼,结构声辐射损耗}和\uline{边界连接阻尼损耗}等三个部分组成。P186 }


\subsection{船舶机械设备工作时将不可避免地引起振动,这种振动包括\textbackslash{}uline\{螺旋桨\}及\textbackslash{}uline\{主机\}引起船体结构振动。P128 }


\subsection{减振器按其性能是否可控又可分为\uline{可调式}和\uline{不可调式}减振器。P261 }


\subsection{在构造上气动减振器主要分为\uline{囊式}和\uline{膜式}. P265 }


\subsection{橡胶金属减振器即是利用\uline{橡胶弹性}及\uline{阻尼耗散}作用达到减振目的的一种减振装置。P261 }


\subsection{引起船体振动的主要振源是\uline{螺旋桨和主机},它们在运转时将引起\uline{周期性干扰力},使船体发生稳态强迫振动。
P128 }


\subsection{在柴油机和螺旋桨干扰力作用下,轴承可能出现扭转、横向和纵向振动,这和\uline{转矩(扭矩),侧向力,弯矩,周期性变化的推力}等有关。
P131 图下面 }


\subsection{按干扰力的频率,螺旋桨干扰力课分为两类:一类是\uline{轴频干扰力},即螺旋桨的干扰频率等于浆轴转速的一阶干扰力;另一类是\uline{叶频或倍叶频干扰力},即干扰频率等于浆轴转速n乘以桨叶数z或桨叶数倍数的高阶干扰力。P128 }


\subsection{表征舰船声隐蔽性的最基本参数是\uline{船舶辐射噪声的声源级}。P191 p3 }


\subsection{随着相对声振动源的距离的增大,其振幅不断减小。原因是\uline{一是部分振动能量被结构吸收,二是散波波前 的扩大}。P190 }


\subsection{吸声材料包括\uline{多孔吸声材料}和\uline{共振吸声材料}。P197 }


\subsection{船舶上层建筑舱室噪声的传播有\uline{空气介质}和\uline{船体结构}两种确定的途径。P192
p4 }


\subsection{对船舶上层建筑舱室的噪声进行预测的比较可行的方法是\uline{灰色预测方法}。P192 p4 }


\subsection{根据舱室噪声的来源,可将舱室噪声分为:\uline{机械噪声、气体流动噪声、脉冲冲击噪声、舰载飞机噪声}等。P286 }


\subsection{次声的防护可从声源、传播途径和接收三方面采取相应的措施进行防护。其中,在传播途径方面,可在次声的传播途径上采取\uline{隔声、吸声、消声}技术。P298}


\subsection{螺旋桨\uline{静力}平衡和\uline{动力}平衡统称为螺旋桨的机械平衡。P129 图旁边 }


\subsection{浮筏隔振系统一般由\uline{机械设备、上层 隔振器、公共筏体、下层隔振器}组成。P268 }


\subsection{海船振动评价衡准包含了\uline{结构强度衡准, 人员舒适性衡准}两部分。P177}


\section{简答}


\subsection{船体总振动的分类及影响总振动的因素?(P111)}

按振动形态,船体总振动可以分为四类
\begin{enumerate}
\item \noindent 在船体的纵中剖面内的垂向弯曲振动,称为垂向振动;
\item \noindent 在船体的水线平面内的水平方向的弯曲振动,称为水平振动;垂向振动和水平振动的振动方向均垂直于船体纵向轴线,故称为横振动。
\item \noindent 船体横剖面绕纵向轴线扭转的振动,称为扭转振动;
\item \noindent 船体横剖面沿其纵向轴线作纵向拉压的往复运动,称为纵向振动。
\end{enumerate}
主振型和主频率由振动体系本身,主要是由船体刚性与船舶质量的分布情况决定的。振幅与干扰力的幅值及系统的刚度有关,对于强迫振动,还与干扰力的频率与系统的固有频率的比值有关。强迫振动的振型取决与这种频率关系及船体振动时的阻尼数值。因此影响总振动的因素有:激振力,阻尼,质量,刚度。有质量和刚度可以推出总振动的模态即频率振型,有激振力和阻尼可以推出船舶响应。


\subsection{结合所学知识,从设计的角度谈一谈如何减少螺旋桨和机舱设备引起的振动和噪声。(各答三条)}

螺旋桨是激起船体振动的一个主要激励源。减小螺旋桨的振动噪声一般原理大致可分为三个方面:首先是改善伴流分布,使之尽可能均匀。在伴流分布以不可能进一步改善时,则可改进螺旋桨设计,减小激励幅频。当上诉两者均达不到理想时,则可在结构上采取措施,减小激励的传递和减小结构响应。 
\begin{enumerate}
\item 改善伴流 ,其方法有改善尾型设计、加装尾鳍、控制去流角等
\item 改进螺旋桨设计 ,方法有增加叶数、采用大侧侧斜和梢部卸载等
\item 减小激励传递,方法有调整间隙、设置避振穴
\end{enumerate}
设备的振动噪声可以通过设计来改善,也可以通过隔振、隔声等方式减少。
\begin{enumerate}
\item 避免共振。改变结构的固有频率或激励频率防止共振的产生。 
\item 减小激励力。进行动平衡或结构改型减小激励幅值。
\item 减小振动或激励力的传递。增加阻尼以防止吸收振动能量,装设减振装置以达到减小幅值的目的。
\end{enumerate}

\subsection{船体产生振动过大的原因可归纳为哪几个方面?(P173)}

船体振动过大可以归结于共振,激励过大和结构设计不良三个方面。

对于共振,可以通过增加频率储备、改变结构固有频率、改变激励频率、改变激励源的作用位置来改善。

对于激励过大,则应考虑减少激励幅值、减少激励传递、减少主机激励来改善。

结构设计不良可以通过合理设计船体结构、船-机-轴-桨的合理配合来改善。


\subsection{船舶减振装置的主要用途有哪些?}

船舶设备减振装置指的是将船舶机械、船舶管路固定到船体承重结构上的弹性连接件上,它包括减振元件本身及中间金属结构。这种船舶减振装置的主要作用如下:
\begin{enumerate}
\item 减振,即减少船舶机械设备工作时传递给船体结构的声振动,降低船舶的水下噪声辐射。
\item 抗冲击,即减少来自船船体的冲击振动对设备的影响。
\item 既减振又抗冲击。
\item 减少设备所受的来自船体支承结构及由螺旋桨旋转产生的强烈低频振动的影响。
\end{enumerate}

\subsection{理想流体介质中波动方程的基本假设(P181)}
\begin{enumerate}
\item 媒质为理想流体;
\item 没有声扰动时,媒质在宏观上是静止的;
\item 声波传播时,媒质中稠密和稀疏的过程是绝热的; 
\item 媒质中传播的是小振幅声波,各声学参量都是一级微量。 
\end{enumerate}

\subsection{有限元法求解声学问题的基本假定(P184)}
\begin{enumerate}
\item 假定流体是可压的,但只允许压力与平均压力相比有较小的变化,流体是各向同性、均匀的。 
\item 声波动过程是绝热的。
\item 假定流体为非流动并且无黏性的(黏性不引起耗散作用)。 
\item 假定流体平均密度和平均压力不变,计算中求解的压力是偏离平均压力的相对压力而不是绝对压力。对模型进行离散,在流体介质中声波应满足波动方程:
\[
\nabla^{^{2}}p=\frac{1}{c^{2}}\frac{\partial^{2}p}{\partial t^{2}}
\]

\end{enumerate}

\subsection{对上层建筑和船体及尾部的耦合分析采取哪4种三维模型。(P166)}
\begin{enumerate}
\item 上层建筑单独从船体主甲板处于船体分离,并加以固定;
\item 将上层建筑与机舱同时考虑,与主船体加以分离,并在机舱前后两段加以固定; 
\item 采用上层建筑—尾部—机舱同时考虑的三维模型,在机舱前段加以固定;
\item 采用全船的三维有限元模型。 
\end{enumerate}

\subsection{橡胶金属减振器,与金属弹簧相比有哪些特征。(答5点)(P261)}

橡胶金属减震器与金属弹簧相比具有以下特征:
\begin{enumerate}
\item 橡胶金属减震器中硫化橡胶的弹性范围非常大,弹性模量较金属材料下降许多。
\item 硫化橡胶形状选择较为自由,可在相当宽的频带范围内对减震器各方向的弹性系数加以调 整,并可获得弯曲、扭曲、翘曲的弹簧作用。 
\item 硫化橡胶的损耗特性远大于金属材料,其 材料的振动减衰性较好,可减小系统共振频带。 
\item 可较容易的得到非线性弹簧特性。
\item 橡胶可与金属牢固粘合,减震装置的安装部分与橡胶能够设计成一个整体,可以获得结构紧 凑的减震装置。 
\item 通过橡胶的柔软性还可减小减震装置与构件结合部分的装配尺寸误差, 具有不开脱的优点。 
\item 耐热、耐寒、耐油等方面比金属弹簧差,因此须注意使用的环境条 件,同时应充分注意橡胶材质的选择。
\end{enumerate}

\subsection{引起上层建筑纵向振动的主要激励有哪些?(P155)}

引起上层建筑纵向振动的主要激励有:
\begin{enumerate}
\item 螺旋桨的叶频激励,通过推力轴承和主船体传递到上层建筑。
\item 由柴油机产生的作 用在曲轴上的径向力,引起轴系的纵向振动,通过推力轴承和主船体传递到上层建筑。
\item 由柴油机产生的作用在曲轴上的切向力,引起曲轴扭转振动,从而引起轴系扭转-纵向振动,通过推力轴承和主船体传递到上层建筑,其激励的主要谐次,对二冲程柴油机为气缸数,对
四冲程柴油机为气缸数或 1/2 气缸数。 
\item 由螺旋桨激励力或轴系纵振与扭振的二次激励引 起机架纵向振动,通过双层底传递到上层建筑,其谐次为气缸数或桨叶数。
\end{enumerate}

\subsection{详细描述预防主机激起振动的措施。(P174)}

预防主机激起振动的措施主要有:
\begin{enumerate}
\item 选择平衡性较好的主机。柴油机是引起船体振动的主要振源之一,在船舶设计阶段就应注意选择较小不平衡惯性力和不平衡惯性力矩的柴油机作为主机,是至关重要的。柴油机缸数越多,其一般平衡性就越好。考虑到与船体低谐调共振的可能性,在船舶设计初期选择主机时,应特别注意主机在常用转速内与船体发生低谐共振一般是不允许的。近年来,随着巨型船舶的发展,主机功率的增大以及尾机型船舶越来越多,就使得主机和螺旋桨两个振源集中在一起,其激振力叠加以后可能增大,
因此要特别注意两个激振力的阶次和相位,否则将引起船体的剧烈振动。 
\item 主机位置。恰当地选择主机位置,可以减小主机引起的船体振动。但必须指,设计阶段确定主机位置的把握程度,是与屎船振动的测试经验和积累资料分不开的。当主机有不平衡惯性力时,应尽可能将主机布置在与船体相对应谐调振动的腹点上;当存在不平衡惯性力矩时,应尽可能将主机布置在与船体相对应谐调振动的腹点上,这样的布置将使主机的激振力不致引起大的振
动。 
\item 安装平衡补偿装置。若已选用了不平衡柴油机作为船舶主机,则应采取有效的平衡 和减震措施,以减小激振力对船体的干扰。\end{enumerate}

\end{document}
