%页面设置
\documentclass[a4paper]{article}
\usepackage[top=1in,bottom=1in,left=1in,right=1in]{geometry}
%\usepackage[utf8]{inputenc}
\usepackage{titlesec} %标题位置,有center,raggedleft,raggedright三个选项
\linespread{1.3}\selectfont
%本地化
\usepackage{fontspec}
\setmainfont{SimSun}
\XeTeXlinebreaklocale "zh"
\XeTeXlinebreakskip = 0pt plus 1pt minus 0.1pt
\usepackage{indentfirst} %段首缩进
\setlength{\parindent}{2em} 
\usepackage{xfrac} %使用sfrac命令
%文档
\begin{document}
\title{船舶噪声控制}
\author{船海 闫鹏 201430110059}
\date{}
\maketitle

船舶机舱中的机电设备是船舶最大的噪声源,处理好机舱噪声对解决船舶噪声有重大意义。机舱的噪声级是由各个机械设备产生的噪声决定的。
\begin{enumerate}
\item 机舱附近的噪声级主要是由它直接辐射的空气噪声;
\item 离机械设备一定距离的噪声级主要是由舱壁反射的空气噪声决定,取决于机舱的混响时间;
\item 相邻舱室中的噪声级主要由机械设备传递的结构噪声决定,同时也于透射过舱壁的空气噪声有关。
\end{enumerate} 
\par 一个噪声系统的主要环节是声源、传输途经和受者。它们之间既有正作用,也有反作用。控制噪声就应当从声源控制、途径控制和受者保护三方面着手。具体采取哪一种或哪几种措施,则应从经济、技术及满足要求等方面综合考虑决定。 
\section{声源控制噪声}
  声源控制噪声是噪声控制中最根本和最有效的手段,选用低噪声的机电设备和对设备作恰当的布置是通常可以采用的方法。
\section{传输途径控制}
传输途径中的控制是最常用的办法,因为一旦机器设计制造和安装完毕,再从生源上控制噪声就受到限制,在传输途径中却容易实现例如隔声、隔振、吸声等都是有效措施,可以起到事后补救的作用。在工厂设计及船舶上层建筑布置中,合理布置可对降低噪声干扰起到重要作用,船舶居住舱室应与机舱等噪声源尽量隔离。使用机罩、消声器等从接近声源处降低噪声,用不同材料使传输途径不连续以控制结构噪声等,都是行之有效的好办法。目前对空气噪声一般采取消声、隔声和吸声处理;而对结构噪声的主要隔声措施是减振、隔振等。
\subsection{吸声/吸振处理}
利用吸声材料和吸声结构来降低室内噪声的降噪技术称为吸声。材料的吸声性能常用吸声系数$\alpha$表示,它是指声波入射到材料表面时,被材料吸收的声能与入射声能之比,与材料表面积一起决定实际吸声能力。一般材料的吸声系数在$0.01一1.00$之间。只有当吸声系数${\alpha}>0.2$的材料才能称为吸声材料。
\par 多孔吸声材料的吸声效果最好,被普遍采用,它分纤维型、泡沫型和颗粒型三种,纤维型多孔吸声材料有玻璃纤维、矿渣棉、毛毡、甘蔗纤维、木丝板等。泡沫型吸声材料有聚氨基甲酯酸泡沫塑料。颗粒型吸声材料有膨胀珍珠岩和微引,吸声砖等。应当注意,吸声材料只吸收反射声,而对声源直接发出的直达声是毫无吸声效果的,因此,当原来房间的吸声效能较高时,如果还用吸声处理来降噪、就不会达到预期的的效果。吸声处理的方法只是在房间不大或原来吸声效果较差的场合才能发挥其减噪作用。 
\par 吸振又叫阻尼降噪,它利用内损耗、内摩擦大的材料作为金属板涂层,减弱噪声在金属板中传播的弯曲波,抑制其振动,从而达到降低辐射噪声的目的。吸振涂层结构一般有刚性吸振涂层、柔性吸振涂层、约束吸振涂层和三明治几类。吸振涂层的性能用损耗系数$\eta$表示。
\subsection{隔声处理}    
利用墙板、门窗、隔声罩等隔声构件将噪声源与受者分隔开来,使噪声在传播途径中受阻以减弱噪声的传递,这种方法称作隔声。它是有效而又应用十分广泛的办法。
\par 按噪声传递方式可分为空气传声(简称空气声)和固体传声(简称固体声)或结构传声两种。空气声指声源直接激发空气振动而产生的声波,并借助空气介质直接传入人耳。固体声是指生源直接激发固体构件振动而发出的声音。固体构件的振动(如锤击地面)以弹性波的形式在墙壁及楼板等构件中传播。在传播中向周围空气辐射发出声波。实际上,声音的传播往往是空气传声和固体传声两者的组合。
\par 对于实心的均匀墙体,其隔声能力决定于墙壁的单位面积重量,其值越大,隔声性能越好。隔声屏用于隔离隔声屏后方身影的噪声,其隔声效果用同一位置上安装隔声屏前后的声能密度之比来表示。隔声罩是抑制机械噪声的较好办法,一般隔声罩采用无孔气密重板,由罩板阻尼涂料和吸声层构成,大型隔声罩的隔声特性总是属于质量控制型,对于某些小型的隔声罩则基本上属于刚度控制型。因此根据噪声的频率设计适合的封闭隔离屏罩很重要。隔声罩的效果采用隔声值dB来衡量。隔声舱室与隔声罩的原理相同,差别仅是前者是防止声能从外部传入,而后者则是防止声能传出。隔声结构一般分为单层和双层。
\subsection{消声器处理}   
消声器是一种控制气流沿管道传播的消声设备。在船舶上,消声器主要用于柴油机的进、排气系统和增压系统,也可用于通风空调设备。消声器应满足消声量大、阻损小、结构性能好三个要求,一般分为三类:阻性消声器、抗性消声器和阻抗复合消声器。阻性消声器结构简单,对中高频噪声吸收能力强,在实际工程中应用广泛,缺点是在高温带水蒸汽即对吸声材料有侵蚀作用的气体中,使用寿命较短,低频噪声消声效果差。通常只用于柴油机进气系统的消声。抗性消声器具有良好的低中频噪声的消声性能,常用的有扩张室消声器和共振腔消声器两大类。复合式消声器能够在较宽的频带范围内取得较高的消声效果,它往往把抗性部分放在前面,阻性部分放在后面或者两者结合在一起。消声器一般设在离排气管末端$\sfrac{1}{4}$管长处,或者直接连接在排气管上。
\section{受着保护}
在机器多而人少(例如船舶机舱),或降低机器噪声不现实或不经济的情况下,对受者的保护是重要手段。人员可以带上护耳器(耳罩或耳塞),防声头盔或在隔声间(如机舱集控室)内值班操作,对一些灵敏仪器(如雷达、微机、电子显微镜、灵敏仪表等),也可用隔声、隔振加以保护。\\
\par 采取噪声控制措施,必须在船舶设计和建造各个阶段认真考虑。使用可靠的噪声预测技术,分析所要选用的各种设备,事先考虑各主要减噪措施,定出最佳的声学设计效果。如航行船舶上噪声污染已经形成,再更换或修改机械设备,所需代价无疑将更大。
\end{document}
