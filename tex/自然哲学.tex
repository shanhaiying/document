%页面设置
\documentclass[a4paper]{article}
\usepackage[top=1in,bottom=1in,left=1.25in,right=1.25in]{geometry}
\usepackage{titlesec} %标题位置,有center,raggedleft,raggedright三个选项
%本地化
\usepackage{fontspec}
\setmainfont{WenQuanYi Zen Hei}
\XeTeXlinebreaklocale "zh"
\XeTeXlinebreakskip = 0pt plus 1pt minus 0.1pt
%文档
\begin{document}
%题头
\title{自然哲学复习题}
\author{闫鹏}
\date{2014.11.16}
\maketitle


\section{如何认识生态自然观和生态文明建设之间的辩证关系?}
\begin{enumerate}
\item 生态自然观是人们面向生态环境问题,依靠生态科学和系统科学,对自然界存在于发展尤其是人与自然界的关系的认识。
\item 生态文明是指人们在改造自然界的同时,通过不断完善的社会制度、改善人的价值观念和思维方式,促进经济、社会和环境协调发展,优化人与自然的关系和人与人关系,建设人与自然和谐统一、协同进化的一种社会文明。
\item 生态自然观是生态文明建设的思想基础,生态自然观主张人是生态系统中的一员,人和生态系统中的其他成员都是平等的,人类不仅要尊重生命共同体中的其他成员,还要善待生命,保护生态环境。这种自然观会影响到人们的思维倾向、思维模式和思维方法,会影响到看待事物的原则、对待生活现实的态度和处理问题的方式,特别是会影响到基本概念和基本规范的形成、理解和运用,成为创建生态文明的思想基础。一、生态自然观为生态文明建设提出的总的要求 二、生态自然观为建立生态技术体系提供了思想基础 三、生态自然观为构建生态政治提供理论指导。
\end{enumerate}

\section{为什么科学发展表现为继承与创新的统一?}

继承:是科学技术发展中的量变,它可使科学知识延续、扩大和加深。科学 是个开放系统,它在时间上有继承性,在空间上有积累性。只有继承已发现的科 学事实、已有理论中的正确东西,科学才能发展、不断完善。 创新:是人类对自然的认识出现新的飞跃,引起科学发展中的质变。创新是 继承的必然趋势和目的,只有在继承的基础上大胆创新,才能不断扩展自然科学知识,提高人类认识和改造自然的能力。科学的发展既有继承又有创新,善于继承才能打好基础,勇于创新才能有所发展,继承是创新的基础和前提,创新是继承的目的和发展,继承是科学发展科学发展连续渐变的过程,创新是科学发展突变飞跃的表现。因此在科学技术的发展模式及动力问题上,马克思主义认为,科学发展在纵向上 表现为渐进与飞跃的统一, 在横向上表现为分化与综合的统一,在总体趋势上表 现为继承与创新的统一。 

\section{多学科的交叉和融贯的方法论意义?}

1、所谓学科交叉方法,就是两门以上的学科之间在面对同一研究对象时, 从不同学科的角度进行对比研究的方法。 2、所谓跨学科方法就是通过多学科的协作共同解决同一问题的方法。亦称多维融贯方法 意义:1.它追求的是片面的深刻之后整体的融合,是单科独立发展与多科学术对话相结合。多学科方法通过相互解释与集体研究消除了片面与偏见,吸收了深刻与 启发意义,实现了高等教育各个学科观点之间的融合。 2.体现了学术宽容与学术规范,发挥了专家见识与集体智慧。学术研究有多元化的权力,有多学科的观点,对不同学科观点不同学者要有学术宽容。多学科研究 方法允许各个学科的富有启发意义的偏见。每位作者可以突出各自领域中在广义的研究方法和观点方面的长处。

\section{如何看待科技对人类异化和对自然异化?}

科技异化实质上是在资本主义制度下劳动异化和人的异化一种必然结果。 由 于劳动是人的最根本最现实的实践活动,是人及人类社会存在的根本方式,劳动 的异化必然带来人的其他社会活动和社会关系的全面异化,科学技术也不例外, 因为“宗教、家庭、国家、法、道德、科学、艺术等等,都不过是生产的一些特 殊的方式,并且受生产的普遍规律的支配。”因此,科学技术作为劳动亦即人处 理自身与自然界关系的社会活动的产物, 也必然随着资本主义社会劳动的异化而 表现出异化的现象。 最根本的是要消灭对科学技术的资本主义利用方式,把现代 科学技术从资本主义制度下解放出来。 也就是说只有通过无产阶级革命来最终解 决资本主义的科技异化问题。当然,在马克思看来,异化的完全克服只有在共产 主义社会制度中才能最终实现。

\section{如何保证科技在社会中健康、持续的运行?}

科学技术的发展和应用要为国家的经济社会发展、长治久安以及可持续发展服务。 科学技术的发展和应用要以人为本,促进民生,推动社会的公平和公正,为和谐社会建设服务。同时科学技术是一把双刃剑。科学技术的运行也带来了一系列的负面影响,有可能产生各种各样风险,如克隆人的伦理风险、水坝和核电站环境风险、转基因食品的健康风险等,引发了一系列争论,造成评价和决策上的困难。在有关科学技术风险公共政策的制定上,应该全面评价科学技术风险一收益的多个方面,批判性地考查“内部”存有争议的科学知识或技术知识,制定恰当的科学技术公共政策。科学技术的发展离不开社会组织系统,社会支持系统,社会环境系统的制约和影响。社会经济基础、社会政治环境和社会文化背景既是影响科学技术发展的重要社会条件,也是影响科学技术发展的社会环境系统中的基本要素。为了科学技术的健康发展, 必须从经济条件、社会环境与国家政策三个方面予以保证。从政策、法规与组织机构,制度化诸方面予以保证,包括建立保障研 发活动社会运行的机制, 建立保障科学技术发展的决策机构,建立适应市场经济 的科学技术体制。

\section{为什么要对科技工作者进行伦理规范?}

一、从社会角度来看,生存在社会当中的每个人都要承担一定的社会责任和伦理责任,科技工作者作为社会的一部分,也必须承担。 并且科技工作者能比一般人更早、更全面、更深刻地了解某一科技活动可能带给人类的危险,他们的能力决定了他们首先应该承担“预见”的伦理责任。  二、科学技术是一把双刃剑,科技的发展在给人类带来福祉的同时,由于对自然界无止境的开发等等,也带来了日益复杂的难题,科技工作者应该更宏观的考虑所做工作带来的利弊。  三、今天的人对未来的人有着无可推卸的责任,有义务为当代人的需求与未来人的生存空间之间把握一个正确的尺度。科技工作者应该谨慎选择有利于人类社会可持续发展的课题进行研究。

\section{中国特色的创新型国家与其它创新型国家有何异同?}

创新型国家是指将科技创新作为国家基本战略,大幅度提高科技创新能力, 形成日益强大竞争优势的国家。中国特色的创新型国家建设的战略任务是在 21 世纪国际科技、 经济竞争日益强化的背景下提出的,国家创新体系的建设是创新 型国家建设的关键。 创新型国家体现了当代科学技术社会一体化的发展趋势, 其特征目前比较公 认的有创新精神、创新人才、创新投入、自主创新能力、创新产出等。 创新型 国家建设的实质是依靠国家的社会管理功能,对国内外创新资源进行有效整合, 不断使科学技术转化为生产力是创新型国家建设的首要和有效路径, 其作法是通 过加大创新投入、增加创新产出、加强自主创新等加快创新型国家建设。 中国特色的国家创新体系是由以企业为主体、产学研结合的技术创新体系、 科学研究与高等教育有机结合的知识创新体系、 军民结合寓军于民的国防科技创 新体系、 各具特色和优势的区域创新体系、社会化和网络化的科技中介服务体系 五个部分构成。 中国特色的创新型国家建设的核心是增强自主创新能力, 建设创新型国家的 总体战略方针是:自主创新、重点跨越、支撑发展、引领未来。

\section{国家创新体系对中国特色的创新型国家建设有何意义?}

国家创新体系是以政府为主导、充分发挥市场配置资源的基础性作用、各类科学技术创新主体紧密联系和有效互动的社会系统。  建设创新型国家,核心是将增强自主创新能力作为发展科学技术的战略基点,把增强自主创新能力作为调整产业结构、转变增长方式的中心环节,将增强自主创新能力作为国家战略,贯穿到现代化建设各个方面。形成有利于自主创新的体制机制,不断巩固和发展中国特色创新型国家建设。我国国家创新体系的构建始于20世纪80年代中期,涉及科技体制革新的经济体制改革。目前我国的国家创新体系建设已取得了显著成绩,从创建国家创新体系到建设创新型国家,并不是单纯的从创新客体向创新主体的转变,而是从整体上考察创新主客体相互作用的结果。完善我国国家创新体系,推进中国特色国家创新体系建设进程,增强我国创新绩效,依靠创新更好驱动经济增长,建设创新型国家。


\section{怎样正确地理解人与自然之间的矛盾和关系,谈谈你对中国可持续发展 之路的想法。}

实现人与自然的和谐关系,是全面、协调、可持续的科学发展观的核心内容之一。在推进经济社会发展的过程中,我们面临人与自然的三重矛盾关系:自然资源和生态环境承载力的有限性与人民日益增长的物质文化需要的矛盾关系;尊重自然的价值与尊重人类发展权利的矛盾关系;技术开发与自然保护的矛盾关系。世界上的任何事物都是矛盾的统一体。我们面对的现实世界,就是由人类社会和自然界双方组成的矛盾统一体,两者之间是辩证统一的关系。一方面,人与自然是相互联系、相互依存、相互渗透的,人类的存在和发展,一刻也离不开自然,必然要通过生产劳动同自然进行物质、能量的交换。与此同时,人与自然之间又是相互对立的。人类为了更好地生存和发展,总是要不断地否定自然界的自然状态,并改变它;而自然界又竭力地否定人,力求恢复到自然状态。人与自然之间这种否定与反否定,改变与反改变的关系,实际上就是作用与反作用的关系,如果对这两种″作用″的关系处理得不好,极易造成人与自然之间失衡。 

中国和世界正处在关键的十字路口。中国的环境恶化很严重,加上庞大的人口和前所未有的经济发展,这些都对中国走向可持续发展形成了重大障碍。随着中国经济的快速发展,资源消耗以及随之产生的废物也大幅度增长,为了取得长期的经济增长,中国必须找到一条可持续发展之路。在我国实施可持续发展必须做好以下几方面工作:首先要改变观念,科学认识自然,掌握自然规律,顺应自然发展,科学地协调、改造自然,善待自然,改变过去那种“先发展,后治理”的老路;其次要珍惜资源,节约资源;最后要唤起公众可持续发展意识,帮助人们树立正确的自然观。中国走可持续发展道路是中国的必然选择,但这条道路同时是十分艰难的,首先经济实力薄弱是一大障碍,其次实现可持续发展需要科学技术特别是高新科学技术的支持,要达到这一点尚需长期努力,最后是地区发展的不平衡,尤其是西部地区水土流失等生态恶化现象更加严重。虽然有上述不足,但我们同时要看到,只要中国政府坚持发挥主导作用,充分运用科技力量,最广泛地动员公众参与,再加上国际社会的有力支持,随着经济体制改革、增长方式转变和科技进步的支持, 中国可持续发展的前景是光明的。

\section{试论人类解决"全球问题"诸如气候问题的前提条件和基本途径。}

全球问题影响到国际社会的安全与稳定,影响到世界人民的生存与发展,影响到人类现代文明建设的进程与前景。可以毫不夸张地说,今天,无论是处理国际经济、政治、军事、文化、社会、科技等关系,还是解决国内的经济发展、政治稳定、社会进步等事务,都难以摆脱全球问题的困扰。然而,由于全球一体化、信息化的洪流来势过猛,全球问题的冲击波过大,所以总的来看,当代人类的回应还颇显欠缺。加强对策研究,制定出范围更广、力度更大的应对措施,开展广泛的国际合作,这些工作无疑非常重要,但前提是提高对全球问题的认识。换言之,只有在对人类已经认同和习惯的一切行为规范、价值准则、理论政策、思维模式进行全面反思的基础上,从文化内涵与理论高度上把握全球问题的真谛,才能面对全球问题的挑战,把人类现代文明建设推向一个新阶段。
全球问题所表现的普遍性、整体性、内在联系的深刻性,要求我们用一种崭新的思维方式认识当代世界,这种思维方式就是全球意识。全球意识要求我们,在承认国际社会存在共同利益,人类文化现象具有共同性的基础上,超越社会制度和意识形态的分歧,克服民族国家和集团利益的限制,以全球的视野去考察、认识社会生活和历史现象。全球意识的立足点有两个:其一,承认人类有共同利益;其二,承认文化有共同性。这两个立足点既是全球意识的核心内容,又是把握全球意识的中心环节。


\section{从科学技术发展历程,说明我们怎样正确理解和评价科学精神及其科学技术的影响和作用?}

科学精神是坚持以科学的态度看待问题、评价问题而不借用非科学或者伪科学的手段。默顿提出了科学精神气质的四原则:普遍主义原则、公有主义原则、无私利性原则、有条理的怀疑主义原则。同时科学精神具有三个方面的层次:
第一层次,即认识活动层面。科学精神的这一层次,包括求实精神、创新精神和怀疑精神。
第二层次,即社会关系层面。科学精神的这一层次,包括宽容精神、竞争精神和诚实精神。
第三层次,即价值关系层次。科学精神的这一层次,包括执着精神、献身精神和全人类精神。 

科学精神对科技的发展所产生的巨大作用可以归结为以下几个方面: 
首先,科学精神是科学事业的灵魂。科学活动是人的活动,人是科学活动的主体,没有人就无所谓科学,而人是受思想观念支配的。这里的人当然是科学探索者,是科学家。科学家在长期的科学探索中,逐步培养了自己高尚的道德情操,磨练了自己坚强的意志,塑造了高贵的品格。这些情操、意志、品格及科学家的行为特征就是科学家的科学精神。没有这种科学家的内在的科学精神,就没有科技的进步和发展。 
其次,科学精神是科学活动的保障。当科学成为一种社会建制后,科学家的探索活动是在科学共同体内进行的,每一个科学家都从属于某一个科学共同体。在科学探索对象日益广泛化、复杂化的条下,“民间科学家”的活动很难在科学上有所发现。科学共同体的活动是现在科学活动的重要形式。但这种形式的活动是在科学制度化的价值观和规范的指导下进行的。而这些制度化的价值观和规范就是科学精神。 
最后,科学精神推定科学革命。在科学发展史上,科学精神不仅使科学摆脱了来自自身以为的束缚(比如神学)而获得了巨大的发展,而且能够促进科学内部因素的变化从而推动科学革命的发生。从古典科学到近代科学、再到现代科学,每一次科学革命的成功都可以说是科学精神的胜利。因为科学精神的核心之一就是创新,科学革命推动了科学的革命和创新。


\section{观察、实验在科学研究中的地位和作用如何? 并述一、两位杰出科学 家具体事例加以说明。}

科学观察是搜集科学事实,获得感性经验的基本途径,是最古老的自然科学研究方法之一。
科学观察是人们为了认识事物的本质和规律,通过感觉器官或辅之以仪器,有目的、有计划地对自然现象在自然发生的条件下进行考察的一种方法。通过科学观察,获取科学事实,从而进行科学研究。科学观察具有自己的特点:1. 科学观察具有目的性和计划性;2.科学观察是在自然发生的条件下对自然现象进行研究;3.科学观察渗透着科学理论。具有客观性,全面性,典型性的原则。
科学观察的作用:科学观察是科学研究的重要环节,科学观察的特点和基本原则表明科学观察在科学研究中的中的重要地位和作用。
1.科学观察是科学认识的来源;2.科学观察是检验科学理论、科学假的重要实现形式;3.科学观察可以导致科学发现,为科学开辟新的研究方向。
科学实验作为一项独立的社会实践活动,它是自然科学发展的重要实践基础;作为一种科学认识活动,它在变革自然中认识自然,比单纯的观察方法具有更加深刻的作用。近代以来,自然科学发展为实验科学,科学实验就成为获取科学事实的一种最基本的方法。
科学实验是人们根据一定的研究目的,运用科学仪器、设备等物质手段,在人为地控制和模拟客观对象的条件下考察对象,从而获取科学事实的一种基本方法。科学实验和科学观察一样,都是科学研究活动中必不可少的认识方法,它们相互依赖,观察依赖实验,实验也离不开观察,在现代科学研究中,观察和实验相结合的整体化趋势也越来越明显。科学实验是在对客观对象实行人为地控制和主动地干预的情况下进行的。就是说,它能够充分地发挥人的主观能动性,揭示自然界发展的内在规律,从而达到对自然界的认识,因而具有不同于单纯的观察的特点。
1.科学实验可以纯化和简化自然现象;2.科学实验可以再现或重演自然过程;3.科学实验可以强化和激化研究对象。
科学实验的作用:1.科学实验是创立科学理论的基础;2.科学实验是检验科学理论的基本手段。

 夜晚的实验
    意大利科学家帕斯拉捷习惯晚饭后到附近的街道上散步。他常常看到,很多蝙蝠灵活的在空中飞来飞去,却从不会撞到墙壁上。这个现象引起了他的好奇:蝙蝠凭什么特殊本领在夜空中自由自在的飞行呢?
    1793年夏天,一个晴朗的夜晚,喧腾热闹的城市渐渐平静下来。帕斯拉捷匆匆吃完饭,便走出街头,把笼子里的蝙蝠放了出去。当他看到放出去的几只蝙蝠轻盈敏捷地来回飞翔时,不由得尖叫起来。因为那几只蝙蝠,眼睛全被他蒙上了,都是“瞎子”呀。
    帕斯拉捷为什么要把蝙蝠的眼睛蒙起来呢?原来,每当他看到蝙蝠在夜晚自由自在的飞翔时,总认为这些小精灵一定长着一双特别敏锐的眼睛,就不可能在黑夜中灵巧的多过各种障碍物,并且敏捷的捕捉飞蛾了。然而事实完全出乎他的意料。帕斯拉捷很奇怪:不用眼睛,蝙蝠凭什么来辨别前方的物体,捕捉灵活的飞蛾呢?
    于是,他把蝙蝠的鼻子堵住.结果,蝙蝠在空中还是飞的那么敏捷、轻松。“难道他薄膜似的翅膀,不仅能够飞翔,而且能在夜间洞察一切吗?”帕斯拉捷这样猜想。他又捉来几只蝙蝠,用油漆涂满它们的全身,然而还是没有影响到它们飞行。
    最后,帕斯拉捷堵住蝙蝠的耳朵,把他们放到夜空中。这次,蝙蝠可没有了先前的神气。他们像无头苍蝇一样在空中东碰西撞,很快就跌落在地。
    啊!蝙蝠在夜间飞行,捕捉食物,原来是靠听觉来辨别方向、确认目标的!
    帕斯拉捷的实验,揭开了蝙蝠飞行的秘密,促使很多人进一步思考:蝙蝠的耳朵又怎么能“穿透”黑夜,“听”到没有声音的物体呢?
    后来人们继续研究,终于弄清了其中的奥秘。原来,蝙蝠靠喉咙发出人耳听不见的“超声波”,这种声音沿着直线传播,一碰到物体就像光照到镜子上那样反射回来。蝙蝠用耳朵接受到这种“超声波”,就能迅速做出判断,灵巧的自由飞翔,捕捉食物。
    现在,人们利用超声波来为飞机、轮船导航,寻找地下的宝藏。超声波就像一位无声的功臣,广泛地应用于工业、农业、医疗和军事等领域。帕斯拉捷怎么也不会想到,自己的实验,会给人类带来如此巨大的恩惠。

波义耳——怀疑派化学家
波义耳十分重视实验研究。他认为只有实验和观察才是科学思维的基础。他总是通过严密的和科学的实验来阐明自己的观点。在物理学方面,他对光的颜色、真空和空气的弹性等进行研究,总结了波义耳气体定律;在化学方面,他对酸、碱和指示剂的研究,对定性检验盐类的方法的探讨,都颇有成效。他是第一位把各种天然植物的汁液用作指示剂的化学家。石蕊试液、石蕊试纸都是他发明的。他还是第一个为酸、碱下了明确定义的化学家,并把物质分为酸、碱、盐三类。他创造了很多定性检验盐类的方法,如利用铜盐溶液是蓝色的,加入氨水溶液变成深蓝色(铜离子与足量氨水形成铜氨络离子)来检验铜盐;利用盐酸和硝酸银溶液混合能产生白色沉淀来检验银盐和盐酸。波义耳的这些发明富有长久的生命力,以至我们今天还经常使用这些最古老的方法。波义耳还在物质成分和纯度的测定、物质的相似性和差异性的研究方面做了不少实验。在1685年发表的《矿泉水的实验研究史的简单回顾》中描述了一套鉴定物质的方法,成为定性分析的先驱。 

\section{全球化形式下的中国科技发展对策研究。}

面对科技全球化的浪潮, 世界各国都在调整科技发展的思路, 我国作为世界上最大的发展中国家, 必须加强对科技全球化的关注和研究, 调整科技发展战略, 把加快科技进步和创新置于经济和社会发展的优先地位。强化科技创新, 特别是原始性创新, 努力实现由 跟踪模仿为主体向以自主创新为主的深刻转变。因此为了迎接科技全球化的挑战, 我国应采取以下对策:
( 一) 调整科技发展战略, 重视科技进步与技术创新
1、加强国家创新体系的建设
2、加强产学研合作, 推动企业技术创新, 提高企业的创新能力
3、组织国家重点技术创新项目, 加快企业 技术进步和产业技术升级
( 二) 积极参与国际科技合作:国际科技合作是我国开放政策的重要组成部分。
( 三) 实施人才战略:人才是科技发展的关键。当今国际间的竞争, 归根到底是人才的竞争。
( 四) 深化科技管理体制, 创新科技管理制度:国内外经验证明, 制度创新是科技进步与创新的必要条件和保障。

\section{论科学伦理和技术伦理的重要性。}

当我们运用马克思主义科学技术观和辩证唯物主义原理,分析科学技术与伦理道德之间的关系,我们看到两者之间既不是等同的,又不是相斥的,更不是不相干的。科学技术与伦理道德是既有区别又有联系的。①科学知识是对客观世界及其规律的正确反映,而道德作为人们行为规范和准则,是对人与人之间伦理关系的反映。它们分属于不同的认识领域,因而社会作用不同。科学用于指导人们改造世界的实践活动,而道德用于调节人与人之间的社会关系。但两者又是密切联系的,都是对客观实际的正确反映,统一于真善美的追求之中;②科学技术与伦理道德也是辩证统一的,从根本上来说,科学技术的发展是人类社会发展的重要推动力,对于伦理道德的发展也是同样具有革命意义的推动力量,表现为科学技术的发展,决定了人类道德前进的基本趋势,促进了新的道德规范的形成,深化了人们的道德认识、更新了人们的道德观念等等。同时进步的社会伦理道德,对科学技术的发展也发挥了重要的精神动力和和文化支撑作用。两者相互制约、相互作用,推动社会向前发展。
因为受到社会经济、政治、文化等其它因素和中间环节的影响和作用,因此相互作用关系不是单向的、直线式的,而是曲折的、复杂的。科学促进伦理的变革,使伦理更好地适应科学和时代的需要;伦理引导科学的进步,使科学更好地为人类造福。科学界也有人担心伦理的规范和引导会不会背离“科学自由”的原则,但实践充分表明,这种担心是多余的。必要的适当的伦理规范非但没有背离科学自由的原则,反而促进了科学顺利健康的发展。

\section{通过生物克隆技术的研究如何理解科学家的道德约束?}

恩格斯曾指出科学技术是推动道德进步的根本力量。根据马克思主义的辩证唯物主义和历史唯物主义原理来分析科技与伦理道德的辩证关系,可以认为,科技与伦理道德之间虽然有一定的联系,但没有必然联系。二者的发展不是完全同步的,高尚的道德和先进的思想,能够引导社会的科学文化向着有利于人类进步的道路顺利发展,而发达的科学技术基础,则有利于社会树立和实现高尚的道德,完整的健全的文明,应该是伦理道德和科技进步的统一。 
科学是一把双刃剑,克隆技术所带来的负面影响已引起愈来愈多的忧患和思考,在享受和期盼当代科技给人类带来幸福和好处的同时,如何建立客观、科学的评价体系便显得特别重要。 
在看待克隆技术的问题上我们应持理性的态度,既不能因噎废食,也不能莽撞冒进。我们不能因克隆技术可能被滥用而禁止它,要禁止的只是对它的滥用,这是社会的责任。为了正当地应用克隆技术,防止对它的滥用,各国应该制定一定的道德规范,颁布实施相关的法律规定,应从法律政策上做一个基本的判断,什么可以研究,什么可以有条件地研究,什么应绝对禁止研究,以及严格规定研究者的法律资格、研究须遵循的法律程序和法律准则,以及违反者的制裁措施。以便充分发挥其潜力,使之服务于社会,同时正确处理克隆技术引起的伦理纠纷。 
因此,要从人与世界、人与自然以及技术与道德等全方位多角度重新审视和确立人类的道德基础,建立科学的道德体系;改变思维方式,给人类文明以更多的关怀,更美好的前程。
最后,我们必须以理性的态度去对待新出现的生命伦理问题,既尊重科学技术,又要尊重人,只有理智地驾驭科学发现和成果,才能使其有效的为人类服务,实现可持续发展。

\section{试述生态自然观的产生及其对辩证唯物主义自然观的丰富和发展}

马克思、恩格斯的生态思想是现代生态自然观的直接的理论来源。在 19 世 纪, 人类的生态环境问题尚没有像现在这样严重,马克思和恩格斯不可能就生态 环境问题进行专门而系统的研究, 但是在他们的理论体系中包含了极其丰富而深 刻的生态思想。生态自然观,是对辩证唯物主义自然观的丰富与发展。生态自然 观确立的现实根源:“生态危机”,生态自然观确立的科学基础:生态科学。 生态自然观的基本思想大体上可以概括为下述几个方面: 其一,生态系统是生命系统。 其二,生态系统具有显著的整体性。 其三,生态系统是自组织的开放系统。 其四,生态系统是动态平衡系统。 其五,生态平衡是稳定性与变化性相统一的平衡。 生态自然观主张把人的角色从大地共同体的征服者改变成共同体的普通成 员与公民, 强调生态系统是一个由相互依赖的各部分组成的共同体,人则是这个 共同体的平等一员和公民, 人类和大自然其他构成者在生态上是平等的;人类不 仅要尊重生命共同体中的其他伙伴,而且要尊重共同体本身;任何一种行为,只有当它有助于保护生命共同体和谐、稳定和美丽时,才是正确的;人与自然之间 要协调发展、共同进化。 

\end{document}
