%页面设置
\documentclass[a4paper]{article}
\usepackage[top=1in,bottom=1in,left=1.25in,right=1.25in]{geometry}
\usepackage{titlesec} %标题位置,有center,raggedleft,raggedright三个选项
%本地化
\usepackage{fontspec}
\setmainfont{微软雅黑}
\XeTeXlinebreaklocale "zh"
\XeTeXlinebreakskip = 0
pt plus 1pt minus 0.1pt
%文档
\begin{document}
\title{第二题}
\author{闫鹏}
\date{}
\maketitle
状态转移矩阵
系统特征方程式为 |rI-A|=\left(\begin{array}{ccc} t - 1 & 0 & 1\\ 0 & t + 2 & 0\\ 1 & 0 & t - 2 \end{array}\right)=t^3 - t^2 - 5\, t + 2=0
可解系统特征值为   -2  0.3820   2.6180,为互异特征值。因此可以代入(rI-A)sigema=0,求出特征向量,将矩阵A化为对角阵。
令sigema=[x1,x2,x3]
当r=-2时,([-2,0,0;0 -2 0;0 0 -2]-[1 0 -1; 0 -2 0; -1 0 2])[x1;x2;x3]=0得
-3x1+x3=0
x1-4x3=0
得x1=x3=0,x2=1.
当r=0.382时,(\left(\begin{array}{ccc} \frac{191}{500} & 0 & 0\\ 0 & \frac{191}{500} & 0\\ 0 & 0 & \frac{191}{500} \end{array}\right)-\left(\begin{array}{ccc} 1 & 0 & -1\\ 0 & -2 & 0\\ -1 & 0 & 2 \end{array}\right))[x1;x2;x3]=0得
0.618x1+x3=0
x2=0
x1-1.618x3=0
令x1=1,得x3=-0.618
当r=2.618时,(\left(\begin{array}{ccc} \frac{1309}{500} & 0 & 0\\ 0 & \frac{1309}{500} & 0\\ 0 & 0 & \frac{1309}{500} \end{array}\right)-\left(\begin{array}{ccc} 1 & 0 & -1\\ 0 & -2 & 0\\ -1 & 0 & 2 \end{array}\right))[x1;x2;x3]=0得
1.618x1+x3=0
x2=0
x1+0.618x3=0
令x1=1,得x3=-1.618
得到矩阵[0 1 1;1 0 0;0 -0.618 -1.618],经过施密特正交化可得最后特征矩阵P=\left(\begin{array}{ccc} 0 & -0.85065 & -0.52573\\ 1.0 & 0 & 0\\ 0 & -0.52573 & 0.85065 \end{array}\right)
P^-1=\left(\begin{array}{ccc} 0 & 1.0 & 0\\ -0.85065 & 0 & -0.52573\\ -0.52573 & 0 & 0.85065 \end{array}\right)
对角阵V=PAP^-1= \left(\begin{array}{ccc} -2.0 & 0 & 0\\ 0 & 0.38197 & 0\\ 0 & 0 & 2.618 \end{array}\right)
通过线性变换法求tao(t)
e^At=P^-1e^VtP= \left(\begin{array}{ccc} 0.27639\, \mathrm{e}^{2.618\, t} + 0.72361\, \mathrm{e}^{0.38197\, t} & 0 & 0.44721\, \mathrm{e}^{0.38197\, t} - 0.44721\, \mathrm{e}^{2.618\, t}\\ 0 & \mathrm{e}^{- 2.0\, t} & 0\\ 0.44721\, \mathrm{e}^{0.38197\, t} - 0.44721\, \mathrm{e}^{2.618\, t} & 0 & 0.72361\, \mathrm{e}^{2.618\, t} + 0.27639\, \mathrm{e}^{0.38197\, t} \end{array}\right)

故tao(t)*x(0)=\left(\begin{array}{c} 1.1708\, \mathrm{e}^{0.38197\, t} - 0.17082\, \mathrm{e}^{2.618\, t}\\ 2\, \mathrm{e}^{- 2.0\, t}\\ 0.27639\, \mathrm{e}^{2.618\, t} + 0.72361\, \mathrm{e}^{0.38197\, t} \end{array}\right)
tao(t-T)Bu(T)=\left(\begin{array}{c} 0.44721\, \mathrm{e}^{0.38197\, t} - 0.44721\, \mathrm{e}^{2.618\, t}\\ 0\\ 0.72361\, \mathrm{e}^{2.618\, t} + 0.27639\, \mathrm{e}^{0.38197\, t} \end{array}\right) %把t变成t-T
积分tao(t-T)Bu(T)=\left(\begin{array}{c} 0.065246\, \mathrm{e}^{2.6181\, t} - 3.0653\, \mathrm{e}^{0.38196\, t} + 3.0\\ \mathrm{e}^{- 2.0\, t} - 1.0\\ 2.0 - 1.8944\, \mathrm{e}^{0.38196\, t} - 0.10557\, \mathrm{e}^{2.6181\, t} \end{array}\right)
xt=expm(A*t)*x0+int(expm(A*(t-tao))*B*1,tao,0,t)=\left(\begin{array}{c} 2.3416\, \mathrm{e}^{0.38197\, t} - 0.34164\, \mathrm{e}^{2.618\, t} - 1.0\\ 2.0\, \mathrm{e}^{- 2.0\, t}\\ 0.55279\, \mathrm{e}^{2.618\, t} + 1.4472\, \mathrm{e}^{0.38197\, t} - 1.0 \end{array}\right)

matlab验证
\begin{verbatim}[matlab]
syms t,tao %定义时间变量t, tao为符号
A=[1 0 -1; 0 -2 0; -1 0 2]; B=[0; 0; 1]; x0=[1; 2; 1] %输入系统状态方程和初始值
xt=expm(A*t)*x0+int(expm(A*(t-tao))*B*1,tao,0,t) %求非齐次解
绘出单位阶跃响应的系统状态轨迹图
t=0:0.1:10
plot(t,xt(1),t,xt(2),t,xt(3))
  %插入图片
\end{verbatim}
\end{document}
